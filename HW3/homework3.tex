\documentclass[twoside,10pt]{article}
\usepackage{amsmath,amsfonts,amsthm,fullpage,amssymb}
%\usepackage{mymath}
\usepackage{algorithm}
\usepackage{algorithmic}
\usepackage{graphicx}


\begin{document}

\title{ISYE 6740 Homework 3}
%\author{Yao Xie}
%\date{Deadline: Feb. 13, Sat., 11:55pm}
\date{100 points total.}
\maketitle


%----------------------------------------------------------------------------------


\begin{enumerate}


\item {\bf Density estimation: Psychological experiments. (50 points)}

 The data set \textsf{n90pol.csv} contains information on 90 university students who participated in a psychological experiment designed to look for relationships between the size of different regions of the brain and political views. The variables \textsf{amygdala} and \textsf{acc} indicate the volume of two particular brain regions known to be involved in emotions and decision-making, the amygdala and the anterior cingulate cortex; more exactly, these are residuals from the predicted volume, after adjusting for height, sex, and similar body-type variables. The variable \textsf{orientation} gives the students' locations on a five-point scale from 1 (very conservative) to 5 (very liberal). %\textsf{orientation} is an ordinal but not a metric variable, so scores of 1 and 2 are not necessarily as far apart as scores of 2 and 3.
 
 \begin{enumerate}
 \item Form 2-dimensional histogram for the pairs of variables (\textsf{amygdala}, \textsf{acc}). Decide on a suitable number of bins so you can see the shape of the distribution clearly. 
 
 \item Now implement kernel-density-estimation (KDE) to estimate the 2-dimensional with a two-dimensional density function of (\textsf{amygdala}, \textsf{acc}). Use the Gaussian kernel \[K(x, y) = \frac{1}{\sqrt {2\pi}} e^{-\frac{(x^2 + y^2)}{2}}.\] Recall the kernel density estimator for a density is given by
 \[
 p(x) = \frac 1 m \sum_{i=1}^m \frac 1 h
 K\left(
 \frac{x^i - x}{h}
 \right)
 \]
 where $h >0$ is the kernel bandwidth. Set an appropriate $h$ so you can see the shape of the distribution clearly. Plot of contour plot (like the ones in slides) for your estimated density. 
 \item Plot the condition distribution of the volume of the \textsf{amygdala} as a function of political \textsf{orientation}: $p(\textsf{amygdala}|\textsf{orientation}=a)$, $a = 1, \ldots, 5$. Do the same for the volume of the 
 \textsf{acc}. Plot $p(\textsf{acc}|\textsf{orientation}=a)$, $a = 1, \ldots, 5$. You may either use histogram or KDE to achieve the goal.
 \end{enumerate}



\item {\bf Implementing EM algorithm for MNIST dataset. (50 points)} 

 Implement the EM algorithm for fitting a Gaussian mixture model for the MNIST dataset. We reduce the dataset to be only two cases, of digits ``2'' and ``6'' only. Thus, you will fit GMM with $C = 2$. Use the data file \textsf{data.mat} or \textsf{data.dat} on Canvas. True label of the data are also provided in \textsf{label.mat} and \textsf{label.dat}

%If you use MATLAB, you may use the following programs (see Canvas) to load the images and their labels:
%\begin{verbatim}
%images = loadMNISTImages('t10k-images-idx3-ubyte');
%\end{verbatim}
The matrix \textsf{images} is of size 784-by-1990, i.e., there are totally 1990 images, and each column of the matrix corresponds to one image of size 28-by-28 pixels (the image is vectorized; the original image can be recovered, e.g., using MATLAB code, \textsf{reshape(images(:,1),28, 28)}.


\begin{enumerate}

\item Select from data one raw image of ``2'' and ``6'' and visualize them, respectively. 

\item Use random Gaussian vector with zero mean as initial means, and identity matrix as initial covariance matrix for the clusters. Please plot the log-likelihood function versus the number of iterations to show your algorithm is converging.

\item  Report the finally fitting GMM model when EM terminates: the weights for each component, the mean vectors (please reformat the vectors into 28-by-28 images and show these images in your submission). Ideally, you should be able to see these means corresponds to ``average'' images.  No need to report the covariance matrices. 

\item (Optional). Use the $p_{ic}$ to infer the labels of the images, and compare with the true labels. Report the miss classification rate.  

%\verbatim{labels = loadMNISTLabels('t10k-labels-idx1-ubyte');}

\end{enumerate}

\end{enumerate}

\end{document}
